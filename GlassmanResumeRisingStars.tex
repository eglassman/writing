% LaTeX file for resume 
% This file uses the resume document class (res.cls)

\documentclass[margin]{res} 
% the margin option causes section titles to appear to the left of body text 
\textwidth=5.2in % increase textwidth to get smaller right margin
%\usepackage{helvetica} % uses helvetica postscript font (download helvetica.sty)
%\usepackage{newcent}   % uses new century schoolbook postscript font 
\usepackage{fancyhdr}
\pagestyle{fancy}
\usepackage{lastpage}
%\setlist[itemize]{leftmargin=*}
\usepackage{enumitem}

\newlabel{LastPage}{{}{4}}  %BAD!! NEEDS TO BE MADE AUTOMATIC!

\begin{document} 

\cfoot{\thepage\ of \pageref{LastPage}}
 
\name{ELENA LEAH GLASSMAN\\[12pt]}
\address{32 Vassar Street, Rm 32-G715\\Cambridge, MA 02139}
%\address{550 Memorial Dr., Apt. 11B-1\\Cambridge, MA 02139}
\address{ELG@MIT.edu\\
(215) 694-9631} 

 
\begin{resume} 
 
\section{Interests} 
Human-computer interaction (HCI), learning at scale, and computer science education. Programming is now being taught at massive scales. I focus on systems for visualizing variation in student solutions to programming problems at scale. I aim to empower teachers with the information they need to assess students' understanding and provide feedback that is relevant to as many students as possible. 

% visualizing and exploring hundreds and thousands of student-written programming solutions

\section{Education} 
Massachusetts Institute of Technology \hfill Cambridge, MA \\
{\bf Ph.D., Electrical Engineering and Computer Science} \hfill Summer 2016\\
4.8/5.0 GPA  \hfill (Expected) \\
Advisor: Robert C. Miller 


Massachusetts Institute of Technology \hfill Cambridge, MA \\
{\bf Master of Eng., Electrical Engineering and Computer Science} \hfill Feb. 2010 \\
Advisor: Russ Tedrake. Thesis: ``A quadratic regulator-based heuristic for rapidly exploring state space.''

Massachusetts Institute of Technology \hfill Cambridge, MA \\
{\bf B.S., Electrical Science and Engineering} \hfill June 2008 \\
4.8/5.0 GPA
 
\section{Journal Articles}

{\bf Elena L. Glassman}, Jeremy Scott, Rishabh Singh, Philip J. Guo, and Robert C. Miller. ``OverCode: Visualizing variation in student solutions to programming problems at scale.'' {\bf Accepted for publication} in the Online Learning at Scale Special Issue of the {\it ACM Transactions on Computer-Human Interaction} (ACM TOCHI), 2015.

{\bf Elena L. Glassman}. ``A wavelet-like filter based on neuron action potentials for analysis of human scalp electroencephalographs.'' \textit{IEEE Transactions on Biomedical Engineering} 52, no. 11 (2005).

\section{Conference\\ Papers}

{\bf Elena L. Glassman}, Ned Gulley, and Robert C. Miller. ``Toward facilitating assistance to students attempting engineering design Problems.'' In {\it Proceedings of the Ninth Annual ACM Conference on International Computing Education Research} (ICER '13). ACM, New York, NY, USA, pp. 41-46, Aug. 2013.

%{\bf Elena L. Glassman}, Ned Gulley, and Robert C. Miller. ``Toward Facilitating Assistance to Students Attempting Engineering Design Problems.'' \textit{Proceedings of the Tenth Annual International Conference on International Computing Education Research}, ICER '13. ACM.

%{\bf Elena L. Glassman}. ``Visualizing and classifying multiple solutions to engineering design problems.'' Extended Abstract. In {\it Proceedings of the Ninth Annual ACM Conference on International Computing Education Research} (ICER '13). ACM, New York, NY, USA, pp. 175-176, Aug. 2013.

%{\bf Elena L. Glassman}. ``Visualizing and classifying multiple solutions to engineering design problems.'' Extended Abstract. Part of the Doctoral Consortium of the \textit{Tenth Annual International Computing Education Research Conference}, ICER '13. ACM.

{\bf Elena L. Glassman}, Alexis Lussier Desbiens, Mark Tobenkin, Mark Cutkosky, and Russ Tedrake. ``Region of Attraction Estimation for a Perching Aircraft: A Lyapunov Method Exploiting Barrier Certificates.'' In {\it Proceedings of the 2012 IEEE International Conference on Robotics and Automation} (ICRA '12), pp. 2235-2242, May 2012.

%{\bf Elena L. Glassman}, Alexis Lussier Desbiens, Mark Tobenkin, Mark Cutkosky, and Russ Tedrake. ``Region of attraction estimation for a perching aircraft: A Lyapunov method exploiting barrier certificates.'' In \textit{Proceedings of the 2012 IEEE International Conference on Robotics and Automation (ICRA)}, 2012.

{\bf Elena L. Glassman} and Russ Tedrake. ``A quadratic regulator-based heuristic for rapidly exploring state space.'' In {\it Proceedings of the 2010 IEEE International Conference on Robotics and Automation} (ICRA '10), pp. 5021-5028, May 2010. 

%{\bf Elena L. Glassman} and Russ Tedrake. ``A quadratic regulator-based heuristic for rapidly exploring state space.'' In \textit{Proceedings of the International Conference on Robotics and Automation (ICRA)}, 2010.


{\bf Elena L. Glassman} and John V. Guttag. ``Reducing the number of channels for an ambulatory patient-specific EEG-based epileptic seizure detector by applying recursive feature elimination.'' In \textit{Proceedings of the 28th Annual International Conference of the IEEE Engineering in Medicine and Biology Society} (EMBS '06), pp. 2175-2178, 30 Aug. - 3 Sept. 2006.


\section{Patent Application}

John V. Guttag, Ali Shoeb, {\bf Elena L. Glassman}, Eugene I. Shih. ``Method and apparatus for reducing the number of channels in an EEG-based epileptic seizure detector.'' US Patent App. 12/196,690, 2008.



\section{Awards and Honors} 

\begin{itemize}[leftmargin=*] %\itemsep -2pt
\item {\bf Amar Bose Teaching Fellowship}, awarded to 3 nominated teaching assistants across MIT \hfill Jan. 2014 - Dec. 2014
\item {\bf NSF Graduate Research Fellowship} \hfill Sept. 2011 - Sept. 2014
              \item {\bf National Defense Science and Engineering Graduate (NDSEG) \\ Fellowship} \hfill Sept. 2008 - Sept. 2011
          
\item {\bf MIT EECS Dept. Masterworks Oral Thesis Presentation Award} \hfill May 2009   
                \item Member, {\bf Eta Kappa Nu}, an EECS honor society \hfill 2008
                 \item {\bf Intel Foundation Young Scientist Award}, given to the top 3 out of 1300 projects at Intel International Science and Engineering Fair \hfill May 2003 
		 \end{itemize}

\section{Research Talks}
{\bf Conference Presentations}
\begin{itemize}[leftmargin=*]
\item {\bf ACM ICER} International Conference on Computing Education Research. ``Toward facilitating assistance to students attempting engineering design problems.'' \hfill August 2013.
\item {\bf IEEE ICRA} International Conference on Robotics and Automation. ``A quadratic regulator-based heuristic for rapidly exploring state space.'' \hfill May 2010.
\end{itemize}
{\bf Seminar Talks}
\begin{itemize}[leftmargin=*]
\item {\bf DUB Seminar, HCI \& Design, University of Washington}. ``OverCode: Visualizing variation in student solutions to programming problems at scale.'' \hfill July 2014.
\end{itemize}
{\bf Posters}
\begin{itemize}[leftmargin=*]
\item {\bf ACM UIST} User Interface Software and Technology Symposium. ``OverCode: Visualizing Variation in Student Solutions to Programming Problems at Scale.'' \hfill October 2014.
\item {\bf ACM Conference on Learning at Scale}. ``Feature engineering for clustering student solutions.'' \hfill March 2014.
\item {\bf ACM ICER} International Conference on Computing Education Research. ``Visualizing and classifying multiple solutions to engineering design problems.'' \hfill August 2013.
\item {\bf IEEE ICRA} International Conference on Robotics and Automation. ``Region of attraction estimation for a perching aircraft: A Lyapunov method exploiting barrier certificates.'' \hfill May 2012.
\end{itemize}
{\bf Doctoral Consortiums}
\begin{itemize}[leftmargin=*]
\item {\bf ACM UIST} User Interface Software and Technology Symposium \hfill October 2014.
\item {\bf ACM ICER} International Conference on Computing Education Research \hfill August 2013.
\end{itemize}
 
\section{Teaching}
%{\bf MIT}
\begin{itemize}[leftmargin=*] %[label={}]
\item {\bf Teaching Assistant, Computation Structures}, {\it MIT} \\Undergraduate lab course on computer architecture. \hfill Spring '12 - Fall '13, Fall '14 \\Ran twice-weekly recitations, created new tools to support students, and assisted students in the course lab space.  
\item {\bf Instructor}, {\it Software Carpentry} \hfill March 2014 \\ 
Center for Urban Science and Progress of the University of New York
\\Worked with a team of instructors to teach a double-room workshop, featuring tracks for Python and R.
\item {\bf Instructor}, {\it Middle East Education through Technology} (MEET) \hfill Summer '13 \\ Jerusalem  \\ Taught the basics of programming and teamwork to Israeli and Palestinian gifted high school sophomores.
\item {\bf Educational video creator}, {\it MIT Teaching and Learning Lab} \hfill Spring '13 \\
Produced for the Singapore University of Technology and Design, explained radio receiver technology.
\item {\bf Instructor, Review of Signals and Systems}, {\it MIT} \hfill January '11, '12, '13 \\Designed and co-taught the EECS Department's month-long course reviewing signals and systems for undergraduate and graduate students.
\item {\bf Teaching Assistant, Introduction to EECS 1}, {\it MIT} \hfill Fall '11 \\ Helped undergraduate students complete their first laboratory in the EECS Department, involving programming, building circuits, and controlling robots.

\item {\bf Tutor, Signals, Systems, \& Probabilistic Systems Analysis}, {\it MIT} \hfill '06 - '11 \\ Assisted students enrolled in EECS courses through the EECS/HKN tutoring service  
\end{itemize}




\section{Training}
\begin{itemize}[leftmargin=*]
\item {\bf Graduate Student Teaching Certificate Program}, {\it MIT} \hfill May '11 \\ A year-long seminar training graduate students in state-of-the-art teaching techniques, run by the MIT Teaching and Learning Lab.
\end{itemize}

\section{Service}
{\bf Leadership}
                \begin{itemize}[leftmargin=*] %\itemsep -2pt
                \item {\bf President}, {\it Middle East Education through Technology's student group at MIT} \\ Serving as an ambassador for the MEET program on campus, and recruiting MIT students as summer instructors \hfill Fall '13 - present
		\item {\bf EdTech Reading Group Co-Organizer}, {\it MIT} \hfill Fall '12 \\ Formed a reading group for MIT students, faculty, and staff to discuss papers relevant to the growing interest in technology in education and education at scale.
                \item {\bf Vice-President}, {\it Eta Kappa Nu, MIT Chapter} \hfill Spring '08 - '09 \\ MIT's EECS honor society 
%\begin{flushright}
%\end{flushright}      
%\item Informally mentor young women wrestlers in Massachusetts
%\item Member of the Board of the Massachusetts Chapter of USA Wrestling
%\item Appointed to be a part-time assistant coach for the new women's wrestling team at Springfield Technical Community College during the Fall '12 season
\end{itemize}

{\bf Program Committees}
 \begin{itemize}[leftmargin=*]
\item ACM Computer-Human Interaction Works-in-Progress ({\bf CHI WiP}) \hfill Jan. '15
\end{itemize}

		 {\bf  Committee Memberships}  
                 \begin{itemize}[leftmargin=*] %\itemsep -2pt
\item {\bf EECS Department Education Committee}, {\it MIT} \hfill Dec. '06 - Fall '08 \\
Served as a student representative during a significant department-wide curriculum redesign.
\item {\bf MIT Council on Educational Technology} \hfill Spring '05 
%\begin{itemize}
%	\item Oversees EECS curriculum
%\end{itemize}
 \end{itemize}
		 


          

		
		 

\section{Research \\Positions}

 {\bf PhD Candidate}, {\it MIT} \hfill Feb '13 - present \\ User Interface Design Group, Computer Science and Artificial Intelligence Lab \\ Cambridge, MA 
 \begin{itemize} \itemsep -2pt  % reduce space between items
 \item Building systems for visualizing and exploring thousands of programming solutions to help teachers more quickly develop a high-level view of students' understanding and misconceptions, and to provide feedback that is relevant to more students.
%\item Developing human-machine systems to support teachers supervising computer-based engineering challenges. %specifically for  for assisting teachers assigning engineering challenges and students.
%\item Developing automated assistance tools for students tackling computer-based engineering challenges.
 \end{itemize}

{\bf Research Intern}, {\it Microsoft Research} \hfill May '14 - Aug. '14 \\ neXus Research Team \\ Redmond, WA 
 \begin{itemize} \itemsep -2pt  % reduce space between items
 \item Created and studied a novel system for classroom use, supervised by Merrie Ringel Morris, Andrés Monroy-Hernández, and Anoop Gupta. 
\end{itemize}

 {\bf Visiting Researcher}, {\it Stanford University} \hfill Fall '10 \\Biomimetics and Dexterous Manipulation Lab  \\ Stanford, CA 
 \begin{itemize} \itemsep -2pt  % reduce space between items
\item As a representative of the MIT Robot Locomotion Group, I collaborated with Stanford University's Biomimetics and Dexterous Manipulation Lab, focusing on control algorithms for future dexterous autonomous aerial vehicles.
 \end{itemize}

 {\bf Graduate Research Assistant}, {\it MIT} \hfill June '08 - May '12 \\ Robot Locomotion Group, Computer Science and Artificial Intelligence Lab \\ Cambridge, MA 
 \begin{itemize} \itemsep -2pt  % reduce space between items
 \item Designed and published optimal control-based distance metrics for use in Rapidly-Exploring Random Trees (RRTs), which can increase the tractability of kinodynamic planning.
 \end{itemize}
 
%  {\bf Undergraduate Researcher} \hfill June '06 - June '08 \\ Robot Locomotion Group, MIT Computer Science and Artificial Intelligence Lab \\ Cambridge, MA 
% \begin{itemize} \itemsep -2pt  % reduce space between items
%\item Created a biomimetic robotic cat that lands on its feet, i.e., flips itself right-side up, when dropped upside down, without net angular momentum.
%	\item Helped design and build the mechanical platform for an ongoing experiment on the control of objects in fluids. 
% \end{itemize}
% 
  {\bf Undergraduate Researcher}, {\it MIT} \hfill Feb. '05 - June '06 \\ Networks \& Mobile Systems Group, Computer Science and Artificial Intelligence Lab \\ Cambridge, MA 
 \begin{itemize} \itemsep -2pt  % reduce space between items
\item Created a data-analysis algorithm for determining the smallest patient-specific subsets of electrodes that still allow an EEG-based epileptic seizure detector to perform at its most accurate level. 
\end{itemize}
 
 
% {\bf 2001-2004 Intel International Science and Engineering Fair Participant} \hfill Sept. '00 - May '04 
% \begin{itemize} \itemsep -2pt  % reduce space between items
%\item \textbf{Brain-Computer Interface for the Muscularly Disabled} (2003, 2004): Designed signal processing method for improved performance of existing BCIs.
%	\item \textbf{Speech Imitation through Analysis, Synthesis, and Optimization} (2002): Software intended as a step towards a speech therapy tool for children.
%	\item \textbf{Signal Dissection by Repetitive Smoothing and Extraction} (2001) 
% \end{itemize}
 


 
%		 
%		{\bf Leadership} 
%                \begin{itemize} \itemsep -2pt
%                \item Vice-President, MIT Chapter of Eta Kappa Nu, an EECS honor society 
%\begin{flushright}
%Apr. '08 - Apr. '09
%\end{flushright}      
%		 \end{itemize}
		
 
		

		 

%		 
%		 				 \section{Technological Public Service}
%\begin{itemize}
%\item Software developer for Assured Labor \hfill Spring 2008
%\begin{itemize}
%	\item Helped produce an early prototype of a cell-phone-centered job marketplace as specified by a start-up hoping to improve the lives of workers in developing countries using mobile phones.
%\end{itemize}
%	\item Member of GlobalHealth, a student-led group focused on improving access to health care in Pakistan \hfill Winter, Spring 2007
%\begin{itemize}
%	\item Helped produce a proposal for a computer analysis-based system for tracking and anticipating outbreaks of various diseases in Pakistan. 
%	\item Proposal was granted funding by the Pakistani government.
%\end{itemize}
%\end{itemize}
 
\section{Selected Press}
\begin{itemize}[leftmargin=*] \itemsep -2pt
\item Appeared in \textit{Science}: ``Rising Stars'' (30 May 2003), \textit{Science} 300 (5624), 1368d.
\item Profiled on CNN's \textit{Lou Dobbs Tonight}, in a segment titled ``America's Bright Future''  \hfill Fall '03 
\item Guest on CNN's \textit{American Morning} \hfill May '03
\end{itemize}

\section{Outreach}
\begin{itemize}[leftmargin=*] \itemsep -2pt
\item Reddit AMA with Jean Yang and Neha Nerula, on behalf of MIT CSAIL \hfill Dec. '14
\item New Hampshire TechFest, Agile robotics booth host \hfill Nov. '11
\item Cambridge Science Festival, Agile robotics booth host \hfill May '11
\item MIT Women's Technology Program \hfill July '08, '11 \\
Guest speaker for a summer program for high school girls interested in EECS
\end{itemize}

\section{Other interests and activities}
{\bf Wrestler}
\begin{itemize}[leftmargin=*] \itemsep -2pt
\item Team Member, MIT's NCAA Div. III Varsity Wrestling Team \hfill Winter '08 - '09
\item Competitor, US and Canada in regional \& national women's tournaments \hfill '09 - '12
\item Two-time Training Camp participant, US Olympic Training Center in Colorago Springs, CO \hfill Aug. '10, Sep '12
\item Board member of the Massachusetts Chapter of USA Wrestling \hfill 2012


\end{itemize}

\end{resume} 
 

\end{document}


