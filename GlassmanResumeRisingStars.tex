% LaTeX file for resume 
% This file uses the resume document class (res.cls)

\documentclass[margin]{res} 
% the margin option causes section titles to appear to the left of body text 
\textwidth=5.2in % increase textwidth to get smaller right margin
%\usepackage{helvetica} % uses helvetica postscript font (download helvetica.sty)
%\usepackage{newcent}   % uses new century schoolbook postscript font 
\usepackage{fancyhdr}
\pagestyle{fancy}
\usepackage{lastpage}

\newlabel{LastPage}{{}{3}}  %BAD!! NEEDS TO BE MADE AUTOMATIC!

\begin{document} 

\cfoot{\thepage\ of \pageref{LastPage}}
 
\name{ELENA LEAH GLASSMAN\\[12pt]}
\address{32 Vassar Street, Rm 32-G715\\Cambridge, MA 02139}
%\address{550 Memorial Dr., Apt. 11B-1\\Cambridge, MA 02139}
\address{ELG@MIT.edu\\
(215) 694-9631} 

 
\begin{resume} 
 
\section{Interests} 
I work on systems for visualizing and exploring thousands of programming solutions that help teachers more quickly develop a high-level view of students' understanding and misconceptions, and to provide feedback that is relevant to more students. 

\section{Keywords} 
Human-computer interaction, information visualization, learning sciences, educational technology, machine learning.

%Intelligent Tutoring Systems, Massive Open Online Courses (MOOCs), and websites like Khan Academy and Codecademy are now used to teach programming courses at a massive scale. A single programming exercise may produce thousands of solutions from learners. Understanding solution variation is important for providing appropriate feedback to students at scale. The wide variation among these solutions can be a source of pedagogically valuable examples, and can be used to refine the autograder for the exercise by exposing corner cases.\\ 
%These systems use both static and dynamic analysis to cluster similar solutions, and let instructors further filter and cluster solutions based on different criteria.I design and carry out user studies to test how these systems affect their target users, namely, teachers and students. We Results indicate that the  allows teachers to more quickly develop a high-level view of students' understanding and misconceptions, and to provide feedback that is relevant to more students.
%My current research focuses on improving the assistance available to students working on computer-based engineering challenges. This work involves providing supplemental information to teachers supervising large numbers of engineering solution submissions, facilitating effective peer-pairing, and automating certain types of debugging assistance for students.
%I have a long-standing interest in the process of learning, creation, and discovery, from an early fascination with creating artificial intelligence to my current focus on the role of play, communities of practice, learning styles, historical context, and storytelling in the development of expertise in a given discipline. My intellectual curiousity is fueled by personal experiences at home, as an MIT undergrad and graduate student, and as an athlete.
% learning how to wrestle well in national and international women's freestyle wrestling tournaments. 
%points to go over: learning for wrestling, communities of practice, results from positive psychology, play, different learning styles, The Talent Code, process of learning rather than what I'm learning, process of science, rather than the conclusions (Bill Bryson's book, Tube, etc.), historical context and storytelling as a part of technical teaching, what-if game
%I am always looking for new, exciting ways for the public to interact with, and be assisted by, technology. I have helped produce an early prototype of a cell-phone-centered job marketplace, specified by Assured Labor, a start-up hoping to improve the lives of workers in developing countries using mobile phones. I also was a team member of GlobalHealth, which attempted to leverage epidemiological data to better distribute Pakistan's limited health care resources in real-time. My earlier work includes proposing the use of socially intelligent robots as speech therapy tools for children, and developing a signal processing method specifically for Brain-Computer Interfaces (BCIs), which allow the muscularly disabled to control computers with their brainwaves, either measured with scalp electrodes or implanted electrode arrays. Now I work on algorithms for making robots more agile; when visitors swing by my lab space, I often hook my robot up to a playstation controller and let them try to control the robot themselves. 
%The control of dynamical systems is not only fundamentally beautiful and challenging, it requires a deep understanding of the natural and/or man-made system to be controlled. The implementation of that control system can require the integration of signal processing, physical actuation, feature selection, and even reinforcement learning. I want to ultimately become a professor working at the intersection of these fields. 

\section{Education} 
Massachusetts Institute of Technology \hfill Cambridge, MA \\
Ph.D., Electrical Engineering and Computer Science \hfill Expected 2015\\
4.8/5.0 GPA (Cumulative Graduate GPA; includes Master's)

Massachusetts Institute of Technology \hfill Cambridge, MA \\
Master of Eng., Electrical Engineering and Computer Science \hfill February 2010

Massachusetts Institute of Technology \hfill Cambridge, MA \\
B.S., Electrical Science and Engineering \hfill June 2008 \\
4.8/5.0 GPA

\section{Internships}

{\bf Research Intern} \hfill May '14 - Aug. '14 \\ neXus Research Team, Microsoft Research \\ Redmond, WA 
 \begin{itemize} \itemsep -2pt  % reduce space between items
 \item Conducting research at the intersection of human-computer interaction (HCI) and education, supervised by Merrie Ringel Morris and Anoop Gupta. 
 \item Focusing on interface for facilitating the production of high quality educational videos using PowerPoint plug-in Office Mix.

\end{itemize}
 
\section{Publications and Patent Applications}

%\begin{itemize}

%\item[] 
Elena L. Glassman, Jeremy Scott, Rishabh Singh, Philip J. Guo, and Robert C. Miller. OverCode: Visualizing Variation in Student Solutions to Programming Problems at Scale. {\bf Submitted for publication} in the Online Learning at Scale Special Issue of the {\it ACM Transactions on Computer-Human Interaction} (ACM TOCHI).

%\item 
Elena L. Glassman, Ned Gulley, and Robert C. Miller. Toward Facilitating Assistance to Students Attempting Engineering Design Problems. In {\it Proceedings of the Ninth Annual ACM Conference on International Computing Education Research} (ICER '13). ACM, New York, NY, USA, p. 41-46, 12-14 Aug. 2013.

%\item Elena L. Glassman, Ned Gulley, and Robert C. Miller. ``Toward Facilitating Assistance to Students Attempting Engineering Design Problems.'' \textit{Proceedings of the Tenth Annual International Conference on International Computing Education Research}, ICER '13. ACM.

%\item 
Elena Glassman. Visualizing and Classifying Multiple Solutions to Engineering Design Problems. Extended Abstract. In {\it Proceedings of the Ninth Annual ACM Conference on International Computing Education Research} (ICER '13). ACM, New York, NY, USA, p. 175-176, 12-14 Aug. 2013.

%\item Elena L. Glassman. ``Visualizing and Classifying Multiple Solutions to Engineering Design Problems.'' Extended Abstract. Part of the Doctoral Consortium of the \textit{Tenth Annual International Computing Education Research Conference}, ICER '13. ACM.

%\item 
Elena Glassman, Alexis Lussier Desbiens, Mark Tobenkin, Mark Cutkosky, and Russ Tedrake. Region of Attraction Estimation for a Perching Aircraft: A Lyapunov Method Exploiting Barrier Certificates, In {\it Proceedings of the 2012 IEEE International Conference on Robotics and Automation} (ICRA '12). p. 2235-2242, 14-18 May 2012.

%\item Elena L. Glassman, Alexis Lussier Desbiens, Mark Tobenkin, Mark Cutkosky, and Russ Tedrake. ``Region of attraction estimation for a perching aircraft: A Lyapunov method exploiting barrier certificates.'' In \textit{Proceedings of the 2012 IEEE International Conference on Robotics and Automation (ICRA)}, 2012.

%\item 
Elena L. Glassman and Russ Tedrake. A quadratic regulator-based heuristic for rapidly exploring state space. In {\it Proceedings of the 2010 IEEE International Conference on Robotics and Automation} (ICRA '10). p. 5021-5028, 3-7 May 2010. 

%\item Elena L. Glassman and Russ Tedrake. ``A quadratic regulator-based heuristic for rapidly exploring state space.'' In \textit{Proceedings of the International Conference on Robotics and Automation (ICRA)}, 2010.
 
%\item 
Elena L. Glassman. A wavelet-like filter based on neuron action potentials for analysis of human scalp electroencephalographs. \textit{IEEE Transactions on Biomedical Engineering} 52, no. 11 (2005).

%\item 
Elena L. Glassman and John V. Guttag. Reducing the number of channels for an ambulatory patient-specific EEG-based epileptic seizure detector by applying recursive feature elimination. In \textit{Proceedings of the 28th Annual International Conference of the IEEE Engineering in Medicine and Biology Society} (EMBS '06). p. 2175-2178, 30 Aug. - 3 Sept. 2006.

%\item 
Elena L. Glassman, John V. Guttag, Eugene I. Shih, and Ali Shoeb. Method and apparatus for reducing the number of channels in an eeg-based epileptic seizure detector. US Patent App. 12/196,690, 2008.

%\item John V. Guttag, Ali Shoeb, Elena L. Glassman, Eugene I. Shih. USPTO Application Number: 20090082689 "Method and Apparatus for reducing the number of channels in an EEG-based epileptic seizure detector."

%\end{itemize}

\section{Appearances in Popular and Scientific Media}
\begin{itemize}
\item Appeared in \textit{Science}: ``Rising Stars'' (30 May 2003), \textit{Science} 300 (5624), 1368d.
\item Profiled on CNN's \textit{Lou Dobbs Tonight}, in a segment titled ``America's Bright Future''  \hfill Fall 2003 
\item Guest on CNN's \textit{American Morning} \hfill May 2003
\end{itemize}
 
\section{Teaching at MIT}
\begin{itemize}
\item Recitation Instructor for Computation Structures \hfill Spring '12 - Fall '13 \\(Undergraduate Lab)
\item Created a short educational video on radio receiver technology for the Singapore University of Technology and Design, funded and produced by the MIT Teaching and Learning Lab \hfill Released Summer '13
\item Teaching Assistant for Introduction to EECS 1 \hfill Fall '11

\item Completed the MIT Teaching and Learning Lab's Graduate Student Teaching Certificate Program
\item Co-taught EECS Dept's Review of Signals and Systems \hfill IAP '11, '12, '13
\item Tutor for Signals and Systems and Probabilistic Systems Analysis through the MIT EECS/HKN tutoring service \hfill '06 - '11
\end{itemize}

\section{Teaching Abroad}
\begin{itemize}
\item Computer Science Instructor for the Middle East Education through Technology Program (MEET) \hfill Jerusalem, Summer '13 
\begin{itemize}
\item Taught the basics of programming and teamwork to Israeli and Palestinian gifted high school sophomores
\end{itemize}
\end{itemize}

\section{Leadership}
                \begin{itemize} \itemsep -2pt
                \item Co-President of the MIT Middle East Education through Technology (MEET) student group, recruiting MIT students as summer instructors \hfill Fall '13 - present
		\item MIT EdTech Reading Group Co-Organizer \hfill Fall '12 
                \item Vice-President, MIT Chapter of Eta Kappa Nu, an EECS honor society 
\begin{flushright}
Apr. '08 - Apr. '09
\end{flushright}      
%\item Informally mentor young women wrestlers in Massachusetts
%\item Member of the Board of the Massachusetts Chapter of USA Wrestling
%\item Appointed to be a part-time assistant coach for the new women's wrestling team at Springfield Technical Community College during the Fall '12 season
\end{itemize}

\section{Professional Activities and Honors} 

               {\bf Fellowships}  
                \begin{itemize} \itemsep -2pt
\item Amar Bose Teaching Fellowship \hfill Jan. '14 - Dec '14
\item NSF Graduate Research Fellowship \hfill Sept. '11 - Sept. '14
              \item National Defense Science and Engineering Graduate Fellowship
\begin{flushright}
Sept. '08 - Sept. '11
\end{flushright}       
          
               \end{itemize}

{\bf Selected Scholarships and Awards}
                   \begin{itemize} \itemsep -2pt
\item MIT EECS Dept. Masterworks Oral Thesis Presentation Award \hfill May 2009   
                
                 \item Intel Foundation Young Scientist Award, given to the top 3 out of 1300 projects at Intel International Science and Engineering Fair \hfill May 2003 
		 \end{itemize}


		 {\bf  Committee Memberships}  
                 \begin{itemize} \itemsep -2pt

               \item    MIT Council on Educational Technology \hfill Spring 2005 
                 
               \item  EECS Department Education Committee \hfill Dec. '06 - Fall '08
%\begin{itemize}
%	\item Oversees EECS curriculum
%\end{itemize}
 \end{itemize}
		 


          

		
		 

\section{Academic Research Positions}

 {\bf Graduate Research Assistant} \hfill Feb '13 - present \\ User Interface Design Group, MIT Computer Science and Artificial Intelligence Lab \\ Cambridge, MA 
 \begin{itemize} \itemsep -2pt  % reduce space between items
 \item Building systems for visualizing and exploring thousands of programming solutions that help teachers more quickly develop a high-level view of students' understanding and misconceptions, and to provide feedback that is relevant to more students.
%\item Developing human-machine systems to support teachers supervising computer-based engineering challenges. %specifically for  for assisting teachers assigning engineering challenges and students.
%\item Developing automated assistance tools for students tackling computer-based engineering challenges.
 \end{itemize}

 {\bf Visiting Researcher} \hfill Fall '10 \\Biomimetics and Dexterous Manipulation Lab, Stanford University  \\ Stanford, CA 
 \begin{itemize} \itemsep -2pt  % reduce space between items
\item As a representative of the MIT Robot Locomotion Group, I collaborated with Stanford University's Biomimetics and Dexterous Manipulation Lab, focusing on control algorithms for future dexterous autonomous aerial vehicles.
 \end{itemize}

 {\bf Graduate Research Assistant} \hfill June '08 - May '12 \\ Robot Locomotion Group, MIT Computer Science and Artificial Intelligence Lab \\ Cambridge, MA 
 \begin{itemize} \itemsep -2pt  % reduce space between items
 \item Designed and published optimal control-based distance metrics for use in Rapidly-Exploring Random Trees (RRTs), which can increase the tractability of kinodynamic planning.
 \end{itemize}
 
%  {\bf Undergraduate Researcher} \hfill June '06 - June '08 \\ Robot Locomotion Group, MIT Computer Science and Artificial Intelligence Lab \\ Cambridge, MA 
% \begin{itemize} \itemsep -2pt  % reduce space between items
%\item Created a biomimetic robotic cat that lands on its feet, i.e., flips itself right-side up, when dropped upside down, without net angular momentum.
%	\item Helped design and build the mechanical platform for an ongoing experiment on the control of objects in fluids. 
% \end{itemize}
% 
  {\bf Undergraduate Researcher} \hfill Feb. '05 - June '06 \\ Networks and Mobile Systems Group, MIT Computer Science and Artificial Intelligence Lab \\ Cambridge, MA 
 \begin{itemize} \itemsep -2pt  % reduce space between items
\item Created a data-analysis algorithm for determining the smallest patient-specific subsets of electrodes that still allow an EEG-based epileptic seizure detector to perform at its most accurate level. 
\end{itemize}
 
 
% {\bf 2001-2004 Intel International Science and Engineering Fair Participant} \hfill Sept. '00 - May '04 
% \begin{itemize} \itemsep -2pt  % reduce space between items
%\item \textbf{Brain-Computer Interface for the Muscularly Disabled} (2003, 2004): Designed signal processing method for improved performance of existing BCIs.
%	\item \textbf{Speech Imitation through Analysis, Synthesis, and Optimization} (2002): Software intended as a step towards a speech therapy tool for children.
%	\item \textbf{Signal Dissection by Repetitive Smoothing and Extraction} (2001) 
% \end{itemize}
 


 
%		 
%		{\bf Leadership} 
%                \begin{itemize} \itemsep -2pt
%                \item Vice-President, MIT Chapter of Eta Kappa Nu, an EECS honor society 
%\begin{flushright}
%Apr. '08 - Apr. '09
%\end{flushright}      
%		 \end{itemize}
		
 
		

		 

%		 
%		 				 \section{Technological Public Service}
%\begin{itemize}
%\item Software developer for Assured Labor \hfill Spring 2008
%\begin{itemize}
%	\item Helped produce an early prototype of a cell-phone-centered job marketplace as specified by a start-up hoping to improve the lives of workers in developing countries using mobile phones.
%\end{itemize}
%	\item Member of GlobalHealth, a student-led group focused on improving access to health care in Pakistan \hfill Winter, Spring 2007
%\begin{itemize}
%	\item Helped produce a proposal for a computer analysis-based system for tracking and anticipating outbreaks of various diseases in Pakistan. 
%	\item Proposal was granted funding by the Pakistani government.
%\end{itemize}
%\end{itemize}
 
\end{resume} 
 

\end{document}


