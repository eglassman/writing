% LaTeX file for resume 
% This file uses the resume document class (res.cls)

\documentclass[margin]{res} 
% the margin option causes section titles to appear to the left of body text 
\textwidth=5.2in % increase textwidth to get smaller right margin
%\usepackage{helvetica} % uses helvetica postscript font (download helvetica.sty)
%\usepackage{newcent}   % uses new century schoolbook postscript font 
\usepackage{fancyhdr}
\pagestyle{fancy}
\usepackage{lastpage}
%\setlist[itemize]{leftmargin=*}
\usepackage{enumitem}

\newlabel{LastPage}{{}{5}}  %BAD!! NEEDS TO BE MADE AUTOMATIC!

\begin{document} 

\cfoot{\thepage\ of \pageref{LastPage}}
 
\name{ELENA LEAH GLASSMAN\\[12pt]}
\address{32 Vassar Street, Rm 32-G707\\Cambridge, MA 02139}
%\address{550 Memorial Dr., Apt. 11B-1\\Cambridge, MA 02139}
\address{ELG@MIT.edu\\
(215) 694-9631} 

 
\begin{resume} 
 
\section{Interests} 
I create tools and user interfaces for teaching and learning online and at scale. My thesis work is focused on tools for teaching programming to thousands of students at once.\\
{\it Human-computer interaction (HCI), learning at scale, computer science education.}

%Thousands of students are learning to program in MOOCs and large residential classes, but we still have the same small staff supporting them. I'm creating user interfaces for (1) giving teachers a bird’s eye view of all the programs students generate and (2) helping students help each other. For example, OverCode helps programming teachers get the gestalt of their hundreds (or thousands) of students' solutions.

%My thesis research is inspired by my experiences as part of the teaching staff for 6.004 Computation Structures, an undergraduate lab class focusing on computer architecture. I continue to serve the course as a recitation instructor and Amar Bose Teaching Fellow.

%I am also a Computer Science instructor ('13) and part of the leadership ('13-'15) for the MIT-MEET program. MEET sends MIT students to Jerusalem to teach both gifted Palestinian and Israeli high school students computer science, entrepreneurship, and teamwork so that they are empowered to create positive change.

%I'm a PhD Candidate in Rob Miller's research group. Before joining Rob Miller's group, I completed my M.Eng. in the MIT CSAIL Robot Locomotion Group. I have also published several papers and share a patent in biomedical signal processing, with the mentorship of John Guttag (MIT CSAIL) and Jack Gelfand.

% visualizing and exploring hundreds and thousands of student-written programming solutions

\section{Education} 
{\bf Massachusetts Institute of Technology} \hfill Cambridge, MA \\
Ph.D., Electrical Engineering and Computer Science \hfill May 2016\\
4.8/5.0 GPA  \hfill (Expected) \\
Advisor: Robert C. Miller 

{\bf Massachusetts Institute of Technology} \hfill Cambridge, MA \\
Master of Eng., Electrical Engineering and Computer Science \hfill Feb. 2010 \\
Advisor: Russ Tedrake. Thesis: ``A quadratic regulator-based heuristic for rapidly exploring state space.''

{\bf Massachusetts Institute of Technology} \hfill Cambridge, MA \\
B.S., Electrical Science and Engineering \hfill June 2008 \\
4.8/5.0 GPA

\section{Research \\Positions}

{\bf User Interface Design Group, MIT CSAIL } \hfill Feb '13 - present \\ 
{\it Ph.D. Candidate} \hfill Cambridge, MA 

{\bf Google} \hfill May '15 - Aug. '15 \\ {\it User Experience Research Intern, Knowledge Graph} \hfill Mountain View, CA 
 \begin{itemize} \itemsep -2pt  % reduce space between items
 \item Prototyping interfaces that help people learn.
\item Mentored by Dan Russell. 
\end{itemize}

{\bf Microsoft Research} \hfill May '14 - Aug. '14 \\ {\it Research Intern, neXus Research Team} \hfill Redmond, WA 
 \begin{itemize} \itemsep -2pt  % reduce space between items
 \item Created, studied, and published Mudslide, a novel system for flipped classrooms.
\item Mentored by Merrie Ringel Morris, Andres Monroy-Hernandez, and Anoop Gupta. 
\end{itemize}

{\bf Stanford University} \hfill Oct. '10 - Jan. '11 \\ {\it Visiting Researcher, Biomimetics and Dexterous Manipulation Lab}  \hfill Stanford, CA 
 \begin{itemize} \itemsep -2pt  % reduce space between items
\item Led an MIT-Stanford collaboration on agile autonomous aerial vehicles, resulting in a publication and a funded grant.
 \end{itemize}

{\bf Robot Locomotion Group, MIT CSAIL} \hfill June '08 - May '12 \\  {\it Graduate Research Assistant} \hfill Cambridge, MA 
 
%  {\bf Undergraduate Researcher} \hfill June '06 - June '08 \\ Robot Locomotion Group, MIT Computer Science and Artificial Intelligence Lab \\ Cambridge, MA 
% \begin{itemize} \itemsep -2pt  % reduce space between items
%\item Created a biomimetic robotic cat that lands on its feet, i.e., flips itself right-side up, when dropped upside down, without net angular momentum.
%	\item Helped design and build the mechanical platform for an ongoing experiment on the control of objects in fluids. 
% \end{itemize}
% 
{\bf Networks \& Mobile Systems Group, MIT CSAIL} \hfill Feb. '05 - June '06 \\{\it Undergraduate Researcher}  \hfill Cambridge, MA 
 \begin{itemize} \itemsep -2pt  % reduce space between items
\item Created and published a novel algorithm for processing EEGs, and later helped file a patent application on the technology.
\end{itemize}

%\newpage

{\bf EEG Lab, Princeton University} \hfill Mar. '04 - Aug. '04 \\ {\it Independent Researcher, invited by the EEG Lab director} \hfill Princeton, NJ 
 
 
\section{Journal Articles}

OverCode: Visualizing variation in student solutions to programming problems at scale.\\
{\bf Elena L. Glassman}, Jeremy Scott, Rishabh Singh, Philip J. Guo, Robert C. Miller. \\ 
{\it ACM Transactions on Computer-Human Interaction} (TOCHI) 22, no. 2 (2015).
\begin{itemize} \itemsep -2pt 
\item Online Learning at Scale Special Issue
\end{itemize}

%\newpage

A wavelet-like filter based on neuron action potentials for analysis of human scalp electroencephalographs.\\ 
{\bf Elena L. Glassman}\\
\textit{IEEE Transactions on Biomedical Engineering} 52, no. 11 (2005).
\begin{itemize} \itemsep -2pt  % reduce space between items
%\item Invited by the research scientist in charge of the lab.
\item A single-author IEEE journal article on the signal processing of EEGs based on my Intel ISEF project, which shared the top award with 2/1300 other projects. 
%\item Joined lab at the invitation of the lab director. 
\end{itemize}

\section{Conference\\ Papers}

Mudslide: A spatially anchored census of student confusion for online lecture videos.\\
{\bf Elena L. Glassman}, Juho Kim, Andres Monroy-Hernandez, Meredith Ringel Morris.\\
CHI 2015: ACM Conference on Human Factors in Computing Systems.\\
\textbf{\emph{Honorable Mention Award (top 5\%)}}{\it (23\% acceptance rate, 10 pages)}

RIMES: Embedding interactive multimedia exercises in lecture videos.\\
Juho Kim, {\bf Elena L. Glassman}, Andres Monroy-Hernandez, Meredith Ringel Morris. \\
CHI 2015: ACM Conference on Human Factors in Computing Systems.\\
{\it (23\% acceptance rate, 10 pages)}

Toward facilitating assistance to students attempting engineering design problems.\\
{\bf Elena L. Glassman}, Ned Gulley, Robert C. Miller. \\
ICER 2013: ACM Conference on International Computing Education Research.\\
{\it (31\% acceptance rate, 6 pages)}

Region of attraction estimation for a perching aircraft: a lyapunov method exploiting barrier certificates.\\
{\bf Elena L. Glassman}, Alexis Lussier Desbiens, Mark Tobenkin, Mark Cutkosky, Russ Tedrake.\\
ICRA 2012: IEEE International Conference on Robotics and Automation.\\
{\it (40\% acceptance rate, 8 pages)}

A quadratic regulator-based heuristic for rapidly exploring state space.\\
{\bf Elena L. Glassman}, Russ Tedrake.\\
ICRA 2010: IEEE International Conference on Robotics and Automation.\\ 
{\it (41\% acceptance rate, 8 pages)}

\section{Technical\\ Reports}

iBCM: Interactive Bayesian Case Model Empowering Humans via Intuitive Interaction.\\
Been Kim, {\bf Elena Glassman}, Brittney Johnson, and Julie Shah.\\
MIT CSAIL TR-2015-010, April 1, 2015.

\section{Posters, Workshops, and Doctoral Consortium Papers}

Learner-Sourcing in an Engineering Class at Scale.\\
{\bf Elena L. Glassman}, Christopher J. Terman, Robert C. Miller.\\
L$@$S 2015: ACM Learning at Scale Conference.

Using and Designing Platforms for In Vivo Educational Experiments.\\
Joseph Jay Williams, Korinn Ostrow, Xi Xiong, {\bf Elena Glasman}, Juho Kim, Samuel Maldonado, Justin
Reich, Neil Heffernan.\\
L$@$S 2015: ACM Learning at Scale Conference.

OverCode: visualizing variation in student solutions to programming problems at scale.\\
{\bf Elena L. Glassman}, Jeremy Scott, Rishabh Singh, Philip J. Guo, Robert C. Miller. \\ 
MIT Big Data Initiative, Nov. 2014.

Interacting with massive numbers of student solutions.\\ (Poster and Doctoral Consortium)\\
{\bf Elena L. Glassman}.\\ 
UIST 2014: ACM User Interface Software and Technology Symposium.

Feature engineering for clustering student solutions.\\
{\bf Elena L. Glassman}, Rishabh Singh, Ned Gulley, Robert C. Miller.\\
CHI 2014: Learning Innovations at Scale Workshop.

Feature engineering for clustering student solutions.\\ 
{\bf Elena L. Glassman}, Rishabh Singh, Robert C. Miller.\\
L$@$S 2014: ACM Learning at Scale Conference.

Mining student-generated alternative implementations.\\
{\bf Elena L. Glassman}, Robert C. Miller.\\
Quanta Workshop and Education Symposium, Taiwan, Jan. 2014.

Visualizing and classifying multiple solutions to engineering design problems.\\
{\bf Elena L. Glassman}.\\ 
ICER 2013: ACM Conference on International Computing Education Research}.

Reducing the number of channels for an ambulatory patient-specific EEG-based epileptic seizure detector by applying recursive feature elimination.\\
{\bf Elena L. Glassman}, John V. Guttag.\\
EMBS 2006: IEEE Engineering in Medicine and Biology Society.

%\section{Patent Application}

%Method and apparatus for reducing the number of channels in an EEG-based epileptic seizure detector. US Patent App. 12/196,690.\\ 
%John V. Guttag, Ali Shoeb, {\bf Elena L. Glassman}, Eugene I. Shih.\\ Cited by 14 other patents and applications.%\\Filed Aug. 2008, published May 2010, denied Aug. 2014.

%\newpage




\section{Awards and Honors} 

\begin{itemize}[leftmargin=*] \itemsep -2pt
\item {\bf Honorable Mention Award} \hfill Apr. '15
\\CHI 2015. Among the top 5\% of all submissions.

\item {\bf Amar Bose Teaching Fellowship} \hfill Jan. '14 - Dec. '14 \\Awarded to 3 nominated teaching assistants across MIT. 

\item {\bf NSF Graduate Research Fellowship} \hfill Sept. '11 - Sept. '14

\item {\bf National Defense Science and Engineering Graduate (NDSEG) \\ Fellowship} \hfill Sept. '08 - Sept. '11
          
\item {\bf MIT EECS Dept. Masterworks Oral Thesis Presentation Award} \hfill May '09   

\item {\bf Eta Kappa Nu}, an EECS honor society \hfill '08

\item {\bf National Gallery for America's Young Inventors} Induction \hfill Feb. '04

\item Selected awards from the {\bf Intel International Science and Engineering Fair}
\begin{itemize}[leftmargin=*] \itemsep -2pt


%\item {\bf Office of Naval Research Award} (\$8,000) \hfill May '04
%\item {\bf First Place in Computer Science} (\$3,000) \hfill May '04

\item {\bf Intel Foundation Young Scientist Award} (\$50,000) \hfill May '03 \\Given to the top 3 out of 1300 projects at Intel International Science and Engineering Fair.
\item {\bf IEEE President's Scholarship} (\$10,000) \hfill May '04
\item {\bf Best of Category: Computer Science} (\$5,000) \hfill May '03
%\item {\bf First Place in Computer Science} (\$3,000) \hfill May '03
%\item {\bf Caltech JPL} Special Achievemant Award (\$1,000) \hfill May '03
%\item {\bf AAAI} Intel Science and Engineering Award \hfill '02, '03, '04
\end{itemize}
\end{itemize}

\section{Selected Press}
\begin{itemize}[leftmargin=*] \itemsep -2pt
\item \textbf{MIT News}: ``Reviewing online homework at scale'' \hfill March '15 \\
Chosen as the MIT homepage Spotlight story 
\item \textbf{The New York Times}: ``Not Too Young for a Patent'' \hfill Feb. '04
\item \textbf{Science}: ``Rising Stars'' (30 May 2003), \textit{Science} 300 (5624), 1368d.
%could add article on falling cat robot in future

\end{itemize}

\section{Profiles, Interviews, and Op-Eds}
\begin{itemize}[leftmargin=*] \itemsep -2pt 
\item {\bf Reddit's Upvoted podcast} \hfill Feb. '15 \\ Interviewed with Jean Yang and Neha Narula.\\
Chosen as one of the A.V. Club's best podcasts of the week.
\item {\bf WIRED} opinion piece: ``MIT Computer Scientists Demonstrate the Hard Way That Gender Still Matters'' with Jean Yang and Neha Narula \hfill Dec. '14
\item Profiled in the {\bf MIT EECS Department Newsletter} \hfill Fall '10
\item \textbf{CNN's Lou Dobbs Tonight}  \hfill Fall '03 \\ Profiled in the segment ``America's Bright Future''
\item \textbf{CNN's American Morning}, Guest \hfill May '03
\end{itemize}

\section{Seminars and Invited Talks}

\begin{itemize}[leftmargin=*] \itemsep -2pt
\item {\bf HarvardX}  \hfill May '15 \\``User Interfaces for Teaching Online and at Scale''

\item {\bf Wellesley HCI}  \hfill March '15 \\``User Interfaces for Teaching Online and at Scale''


\item {\bf DUB Seminar, HCI \& Design, U. of Washington} \hfill July '14 \\``OverCode: Visualizing variation in student solutions to programming problems at scale.''

\item {\bf Schlumberger-Doll Research Center} \hfill Oct. '01 \\``Signal Dissection by Repetitive Smoothing and Extraction.'' \\
Talk given as part of receiving the Schlumberger Excellence in Educational Development award at Intel ISEF 2001.
\end{itemize}


\section{Public Speaking}
\begin{itemize}[leftmargin=*] \itemsep -2pt 
\item Panelist, Women Techmaker's Summit at Google Cambridge \hfill March '15
\item Invited speaker, MIT CSAIL's {\bf Hour of Code} event \hfill Dec. '14
\item Panelist, MIT EECS Teaching Assistant Orientation \hfill Feb. '13
\item Invited speaker, MIT Women's Technology Program \hfill July '08, '11
\item Invited speaker, MIT CSAIL Campus Preview Weekend \hfill Apr. '08
\end{itemize}


 
\section{Teaching}
%{\bf MIT}
\begin{itemize}[leftmargin=*] %[label={}]
\item {\bf Teaching Assistant, Computation Structures}, {\it MIT} \\Undergraduate lab course on computer architecture. \hfill Spring '12 - Fall '13, Fall '14 \\Ran twice-weekly recitations, created new tools to support students, and assisted students in the course lab space.  
\item {\bf Instructor}, {\it Software Carpentry, NYU} \hfill Mar. '14  
%Center for Urban Science and Progress of the University of New York
\\Worked with a team of instructors to teach a workshop covering Python and git.
\item {\bf Instructor}, {\it Middle East Education through Technology} (MEET) \hfill Summer '13 \\ Taught the basics of programming and teamwork to Israeli and Palestinian gifted high school sophomores in Jerusalem.
\item {\bf Educational video creator}, {\it MIT Teaching and Learning Lab} \hfill Spring '13 \\
Produced for the Singapore University of Technology and Design, explained radio receiver technology.
\item {\bf Instructor, Review of Signals \& Systems}, {\it MIT} \hfill Jan. '11, '12, '13 %\\Designed and co-taught the EECS Department's month-long course reviewing signals and systems for undergraduate and graduate students.
\item {\bf Teaching Assistant, Introduction to EECS 1}, {\it MIT} \hfill Fall '11 %\\ Helped undergraduate students complete their first laboratory in the EECS Department, involving programming, building circuits, and controlling robots.

\item {\bf Tutor, Signals, Systems, \& Probabilistic Systems Analysis}, {\it MIT} \hfill '06 - '11 %\\ Assisted students enrolled in EECS courses through the EECS/HKN tutoring service  
\end{itemize}

\section{Research Mentoring}
\begin{itemize}[leftmargin=*] 
\item Stacey Terman, MIT undergraduate\\
Master's thesis proposal
\item Aaron Lin, MIT undergraduate\\
Built and deployed Dear Beta, a platform for crowdsourcing hints in a large undergraduate computer architecture course
\end{itemize}

\section{Training}
\begin{itemize}[leftmargin=*]
\item {\bf Graduate Student Teaching Certificate Program}, {\it MIT} \hfill May '11 \\ A year-long seminar in state-of-the-art teaching techniques.
%\\ A year-long seminar training graduate students in state-of-the-art teaching techniques, run by the MIT Teaching and Learning Lab.
\end{itemize}

\section{Service and Leadership}

                \begin{itemize}[leftmargin=*] %\itemsep -2pt
\item {\bf Reviewer}, {\it User Interface Software and Technology} (UIST) \hfill May '15 
\item {\bf Session Chair} {\it ACM Computer-Human Interaction} (CHI) \hfill Apr. '15 \\
Social Media \& Citizen Science
\item {\bf Works-in-Progress Program Committee} {\it ACM Computer-Human Interaction} (CHI)  \hfill Jan. '15 


\item {\bf President}, {\it Middle East Education through Technology's student group at MIT} \\ Recruiting and coordinating MIT students as summer instructors. \hfill Fall '13 - present %Serving as an ambassador for the MEET program on campus, and recruiting MIT students as summer instructors. 

		\item {\bf MIT EdTech Reading Group Co-Organizer} \hfill Fall '12 \\ Formed a reading group for MIT students, faculty, and staff to discuss papers relevant to the growing interest in technology in education and education at scale.

                \item {\bf Eta Kappa Nu Vice-President}, {\it MIT Chapter} \hfill Spring '08 - '09 \\ MIT's EECS honor society 


\item {\bf MIT EECS Department Education Committee} \hfill Dec. '06 - Fall '08 \\
Served as a student representative during a significant department-wide curriculum redesign.
\item {\bf MIT Council on Educational Technology} \hfill Spring '05 


 \end{itemize}
		 


          

		
		 


 
 
% {\bf 2001-2004 Intel International Science and Engineering Fair Participant} \hfill Sept. '00 - May '04 
% \begin{itemize} \itemsep -2pt  % reduce space between items
%\item \textbf{Brain-Computer Interface for the Muscularly Disabled} (2003, 2004): Designed signal processing method for improved performance of existing BCIs.
%	\item \textbf{Speech Imitation through Analysis, Synthesis, and Optimization} (2002): Software intended as a step towards a speech therapy tool for children.
%	\item \textbf{Signal Dissection by Repetitive Smoothing and Extraction} (2001) 
% \end{itemize}
 


 
%		 
%		{\bf Leadership} 
%                \begin{itemize} \itemsep -2pt
%                \item Vice-President, MIT Chapter of Eta Kappa Nu, an EECS honor society 
%\begin{flushright}
%Apr. '08 - Apr. '09
%\end{flushright}      
%		 \end{itemize}
		
 
		

		 

%		 
%		 				 \section{Technological Public Service}
%\begin{itemize}
%\item Software developer for Assured Labor \hfill Spring 2008
%\begin{itemize}
%	\item Helped produce an early prototype of a cell-phone-centered job marketplace as specified by a start-up hoping to improve the lives of workers in developing countries using mobile phones.
%\end{itemize}
%	\item Member of GlobalHealth, a student-led group focused on improving access to health care in Pakistan \hfill Winter, Spring 2007
%\begin{itemize}
%	\item Helped produce a proposal for a computer analysis-based system for tracking and anticipating outbreaks of various diseases in Pakistan. 
%	\item Proposal was granted funding by the Pakistani government.
%\end{itemize}
%\end{itemize}
 




\section{Outreach}
\begin{itemize}[leftmargin=*] \itemsep -2pt
\item {\bf Reddit AMA} on gender, CS, and academia with Jean Yang and Neha Nerula  \\
Received 4763 comments, rose to the top 5 stories on the Reddit homepage, and was covered by Business Insider, Gigaom, and BostInno among others. \hfill Dec. '14
\item {\bf Harvard Women in CS}'s ``Women Engineers Code Hackathon'', Mentor \hfill Dec. '13
\item {\bf Cambridge Science Festival}, Robotics representative \hfill Nov. '11
\item {\bf NH TechFest}, Robotics representative \hfill May '11
%\item Cambridge Science Festival, agile robotics representative \hfill May '11

%Guest spoke twice to gifted high school girls interested in EECS
\end{itemize}

\section{Other activities}
{\bf Wrestling}
\begin{itemize}[leftmargin=*] \itemsep -2pt
\item Team Member, MIT's NCAA Div. III Varsity Wrestling Team \hfill Winter '08 - '09
\item Competitor, US and Canada in regional \& national women's tournaments \hfill '09 - '12
\item Two-time Training Camp participant, US Olympic Training Center in Colorago Springs, CO \hfill Aug. '10, Sept. '12
%\item Board member of the Massachusetts Chapter of USA Wrestling \hfill 2012


\end{itemize}

\end{resume} 
 

\end{document}


