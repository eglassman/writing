% LaTeX file for resume 
% This file uses the resume document class (res.cls)

\documentclass[margin]{res} 
% the margin option causes section titles to appear to the left of body text 
\textwidth=5.2in % increase textwidth to get smaller right margin
%\usepackage{helvetica} % uses helvetica postscript font (download helvetica.sty)
%\usepackage{newcent}   % uses new century schoolbook postscript font 
\usepackage{fancyhdr}
\pagestyle{fancy}
\usepackage{lastpage}
%\setlist[itemize]{leftmargin=*}
\usepackage{enumitem}

\newlabel{LastPage}{{}{5}}  %BAD!! NEEDS TO BE MADE AUTOMATIC!

\begin{document} 

\cfoot{\thepage\ of \pageref{LastPage}}
 
\name{ELENA LEAH GLASSMAN\\[12pt]}
\address{32 Vassar Street, Rm 32-G715\\Cambridge, MA 02139}
%\address{550 Memorial Dr., Apt. 11B-1\\Cambridge, MA 02139}
\address{ELG@MIT.edu\\
(215) 694-9631} 

 
\begin{resume} 
 
\section{Interests} 
I create tools for teaching programming to thousands of students at once. Specifically, I focus on systems for visualizing variation in student solutions to programming problems at scale. I aim to empower teachers with the information they need to assess students' understanding and provide feedback that is relevant to as many students as possible. I also am president of MIT-MEET, which helps teach gifted Israelis and Palestinians computer science and teamwork in Jerusalem.\\
{\it Human-computer interaction (HCI), learning at scale, computer science education.}

%Thousands of students are learning to program in MOOCs and large residential classes, but we still have the same small staff supporting them. I'm creating user interfaces for (1) giving teachers a bird’s eye view of all the programs students generate and (2) helping students help each other. For example, OverCode helps programming teachers get the gestalt of their hundreds (or thousands) of students' solutions.

%My thesis research is inspired by my experiences as part of the teaching staff for 6.004 Computation Structures, an undergraduate lab class focusing on computer architecture. I continue to serve the course as a recitation instructor and Amar Bose Teaching Fellow.

%I am also a Computer Science instructor ('13) and part of the leadership ('13-'15) for the MIT-MEET program. MEET sends MIT students to Jerusalem to teach both gifted Palestinian and Israeli high school students computer science, entrepreneurship, and teamwork so that they are empowered to create positive change.

%I'm a PhD Candidate in Rob Miller's research group. Before joining Rob Miller's group, I completed my M.Eng. in the MIT CSAIL Robot Locomotion Group. I have also published several papers and share a patent in biomedical signal processing, with the mentorship of John Guttag (MIT CSAIL) and Jack Gelfand.

% visualizing and exploring hundreds and thousands of student-written programming solutions

\section{Education} 
Massachusetts Institute of Technology \hfill Cambridge, MA \\
{\bf Ph.D., Electrical Engineering and Computer Science} \hfill Summer 2016\\
4.8/5.0 GPA  \hfill (Expected) \\
Advisor: Robert C. Miller 


Massachusetts Institute of Technology \hfill Cambridge, MA \\
{\bf Master of Eng., Electrical Engineering and Computer Science} \hfill Feb. 2010 \\
Advisor: Russ Tedrake. Thesis: ``A quadratic regulator-based heuristic for rapidly exploring state space.''

Massachusetts Institute of Technology \hfill Cambridge, MA \\
{\bf B.S., Electrical Science and Engineering} \hfill June 2008 \\
4.8/5.0 GPA

\section{Research \\Positions}

 {\bf Ph.D. Candidate}, {\it MIT} \hfill Feb '13 - present \\ User Interface Design Group, Computer Science and Artificial Intelligence Lab \\ Cambridge, MA 


{\bf Research Intern}, {\it Microsoft Research} \hfill May '14 - Aug. '14 \\ neXus Research Team \\ Redmond, WA 
 \begin{itemize} \itemsep -2pt  % reduce space between items
 \item Created, studied, and published a novel system for classroom use, supervised by Merrie Ringel Morris, Andres Monroy-Hernandez, and Anoop Gupta. 
\end{itemize}

 {\bf Visiting Researcher}, {\it Stanford University} \hfill Fall '10 \\Biomimetics and Dexterous Manipulation Lab  \\ Stanford, CA 
 \begin{itemize} \itemsep -2pt  % reduce space between items
\item Led an MIT-Stanford collaboration on agile autonomous aerial vehicles, resulting in a publication and a funded grant.
 \end{itemize}

 {\bf Graduate Research Assistant}, {\it MIT} \hfill June '08 - May '12 \\ Robot Locomotion Group, Computer Science and Artificial Intelligence Lab \\ Cambridge, MA 
 
%  {\bf Undergraduate Researcher} \hfill June '06 - June '08 \\ Robot Locomotion Group, MIT Computer Science and Artificial Intelligence Lab \\ Cambridge, MA 
% \begin{itemize} \itemsep -2pt  % reduce space between items
%\item Created a biomimetic robotic cat that lands on its feet, i.e., flips itself right-side up, when dropped upside down, without net angular momentum.
%	\item Helped design and build the mechanical platform for an ongoing experiment on the control of objects in fluids. 
% \end{itemize}
% 
  {\bf Undergraduate Researcher}, {\it MIT} \hfill Feb. '05 - June '06 \\ Networks \& Mobile Systems Group, Computer Science and Artificial Intelligence Lab \\ Cambridge, MA 
 \begin{itemize} \itemsep -2pt  % reduce space between items
\item Created and published a novel algorithm for processing EEGs, and later helped file a patent on the technology.
\end{itemize}

\newpage

  {\bf Volunteer Researcher}, {\it Princeton University} \hfill Mar. '04 - Aug. '04 \\ EEG Lab, Princeton Neuroscience Institute \\ Princeton, NJ 
 \begin{itemize} \itemsep -2pt  % reduce space between items
\item Composed a single-author IEEE journal article on the signal processing of EEGs based on my Intel ISEF project, which shared the top award with 2 other projects out of 1300.
\end{itemize}
 
\section{Journal Articles}

OverCode: Visualizing variation in student solutions to programming problems at scale.\\
{\bf Elena L. Glassman}, Jeremy Scott, Rishabh Singh, Philip J. Guo, Robert C. Miller. \\ 
{\it ACM Transactions on Computer-Human Interaction} (ACM TOCHI)\\
{\it Accepted for publication in the Online Learning at Scale Special Issue, 2015.}

A wavelet-like filter based on neuron action potentials for analysis of human scalp electroencephalographs.\\ 
{\bf Elena L. Glassman}\\
\textit{IEEE Transactions on Biomedical Engineering} 52, no. 11 (2005).

\section{Conference\\ Papers}

Mudslide: A spatially anchored census of student confusion for online lecture videos.\\
{\bf Elena L. Glassman}, Juho Kim, Andres Monroy-Hernandez, Meredith Ringel Morris.\\
CHI 2015: ACM Conference on Human Factors in Computing Systems.\\
{\it Conditionally accepted. (23\% acceptance rate, 10 pages)}

RIMES: Embedding interactive multimedia exercises in lecture videos.\\
Juho Kim, {\bf Elena L. Glassman}, Andres Monroy-Hernandez, Meredith Ringel Morris. \\
CHI 2015: ACM Conference on Human Factors in Computing Systems.\\
{\it Conditionally accepted. (23\% acceptance rate, 10 pages)}

Toward facilitating assistance to students attempting engineering design problems.\\
{\bf Elena L. Glassman}, Ned Gulley, Robert C. Miller. \\
ICER 2013: ACM Conference on International Computing Education Research.

Region of attraction estimation for a perching aircraft: a lyapunov method exploiting barrier certificates.\\
{\bf Elena L. Glassman}, Alexis Lussier Desbiens, Mark Tobenkin, Mark Cutkosky, Russ Tedrake.\\
ICRA 2012: IEEE International Conference on Robotics and Automation.

A quadratic regulator-based heuristic for rapidly exploring state space.\\
{\bf Elena L. Glassman}, Russ Tedrake.\\
ICRA 2010: IEEE International Conference on Robotics and Automation. 

Reducing the number of channels for an ambulatory patient-specific EEG-based epileptic seizure detector by applying recursive feature elimination.\\
{\bf Elena L. Glassman}, John V. Guttag.\\
EMBS 2006: IEEE Engineering in Medicine and Biology Society.


\section{Patent Application}

Method and apparatus for reducing the number of channels in an EEG-based epileptic seizure detector. US Patent App. 12/196,690. \\
John V. Guttag, Ali Shoeb, {\bf Elena L. Glassman}, Eugene I. Shih.\\
Cited by 14 other patents and applications.\\
Filed Aug. 2008, published May 2010, denied Aug. 2014.

\newpage

\section{Posters, Workshops, and Doctoral Consortium Papers}

OverCode: visualizing variation in student solutions to programming problems at scale.\\
{\bf Elena L. Glassman}, Jeremy Scott, Rishabh Singh, Philip J. Guo, Robert C. Miller. \\ 
MIT Big Data Initiative, Nov. 2014.

Interacting with massive numbers of student solutions.\\ (Poster and Doctoral Consortium)\\
{\bf Elena L. Glassman}.\\ 
UIST 2014: ACM User Interface Software and Technology Symposium.

Feature engineering for clustering student solutions.\\
{\bf Elena L. Glassman}, Rishabh Singh, Ned Gulley, Robert C. Miller.\\
CHI 2014: Learning Innovations at Scale Workshop.

Feature engineering for clustering student solutions.\\ 
{\bf Elena L. Glassman}, Rishabh Singh, Robert C. Miller.\\
L$@$S 2014: ACM Learning at Scale Conference.

Mining student-generated alternative implementations.\\
{\bf Elena L. Glassman}, Robert C. Miller.\\
Quanta Workshop and Education Symposium, Taiwan, Jan. 2014.

Visualizing and classifying multiple solutions to engineering design problems.\\
{\bf Elena L. Glassman}.\\ 
ICER 2013: ACM Conference on International Computing Education Research}.


\section{Awards and Honors} 

\begin{itemize}[leftmargin=*] %\itemsep -2pt
\item {\bf Amar Bose Teaching Fellowship}, awarded to 3 nominated teaching assistants across MIT \hfill Jan. 2014 - Dec. 2014
\item {\bf NSF Graduate Research Fellowship} \hfill Sept. 2011 - Sept. 2014
              \item {\bf National Defense Science and Engineering Graduate (NDSEG) \\ Fellowship} \hfill Sept. 2008 - Sept. 2011
          
\item {\bf MIT EECS Dept. Masterworks Oral Thesis Presentation Award} \hfill May 2009   
                \item Inducted into {\bf Eta Kappa Nu}, an EECS honor society \hfill 2008
                 \item {\bf Intel Foundation Young Scientist Award}, given to the top 3 out of 1300 projects at Intel International Science and Engineering Fair \hfill May 2003 
		 \end{itemize}

\section{Seminars}
%{\bf Conference Presentations}
%\begin{itemize}[leftmargin=*]
%\item {\bf ACM ICER} International Conference on Computing Education Research. ``Toward facilitating assistance to students attempting engineering design problems.'' \hfill August 2013.
%\item {\bf IEEE ICRA} International Conference on Robotics and Automation. ``A quadratic regulator-based heuristic for rapidly exploring state space.'' \hfill May 2010.
%\end{itemize}
%{\bf Seminars}
\begin{itemize}[leftmargin=*]
\item {\bf DUB Seminar, HCI \& Design, University of Washington}. ``OverCode: Visualizing variation in student solutions to programming problems at scale.'' \hfill July 2014.
\item {\bf Invited Talk, Schlumberger-Doll Research Center in Ridgefield, Connecticut.} ``Signal Dissection by Repetitive Smoothing and Extraction.'' \hfill Oct. 2001.
Given as part of receiving the Schlumberger Excellence in Educational Development at Intel ISEF 2001.
\end{itemize}
%{\bf Posters}
%\begin{itemize}[leftmargin=*]
%\item {\bf ACM UIST} User Interface Software and Technology Symposium. ``OverCode: Visualizing Variation in Student Solutions to Programming Problems at Scale.'' \hfill October 2014.
%\item {\bf ACM Conference on Learning at Scale}. ``Feature engineering for clustering student solutions.'' \hfill March 2014.
%\item {\bf ACM ICER} International Conference on Computing Education Research. ``Visualizing and classifying multiple solutions to engineering design problems.'' \hfill August 2013.
%\item {\bf IEEE ICRA} International Conference on Robotics and Automation. ``Region of attraction estimation for a perching aircraft: A Lyapunov method exploiting barrier certificates.'' \hfill May 2012.
%\end{itemize}
%{\bf Doctoral Consortiums}
%\begin{itemize}[leftmargin=*]
%\item {\bf ACM UIST} User Interface Software and Technology Symposium \hfill October 2014.
%\item {\bf ACM ICER} International Conference on Computing Education Research \hfill August 2013.
%\end{itemize}
 
\section{Teaching}
%{\bf MIT}
\begin{itemize}[leftmargin=*] %[label={}]
\item {\bf Teaching Assistant, Computation Structures}, {\it MIT} \\Undergraduate lab course on computer architecture. \hfill Spring '12 - Fall '13, Fall '14 \\Ran twice-weekly recitations, created new tools to support students, and assisted students in the course lab space.  
\item {\bf Instructor}, {\it Software Carpentry} \hfill March 2014 \\ 
Center for Urban Science and Progress of the University of New York
\\Worked with a team of instructors to teach a double-room workshop, featuring tracks for Python and R.
\item {\bf Instructor}, {\it Middle East Education through Technology} (MEET) \hfill Summer '13 \\ Jerusalem  \\ Taught the basics of programming and teamwork to Israeli and Palestinian gifted high school sophomores.
\item {\bf Educational video creator}, {\it MIT Teaching and Learning Lab} \hfill Spring '13 \\
Produced for the Singapore University of Technology and Design, explained radio receiver technology.
\item {\bf Instructor, Review of Signals and Systems}, {\it MIT} \hfill January '11, '12, '13 \\Designed and co-taught the EECS Department's month-long course reviewing signals and systems for undergraduate and graduate students.
\item {\bf Teaching Assistant, Introduction to EECS 1}, {\it MIT} \hfill Fall '11 \\ Helped undergraduate students complete their first laboratory in the EECS Department, involving programming, building circuits, and controlling robots.

\item {\bf Tutor, Signals, Systems, \& Probabilistic Systems Analysis}, {\it MIT} \hfill '06 - '11 \\ Assisted students enrolled in EECS courses through the EECS/HKN tutoring service  
\end{itemize}




\section{Training}
\begin{itemize}[leftmargin=*]
\item {\bf Graduate Student Teaching Certificate Program}, {\it MIT} \hfill May '11 \\ A year-long seminar training graduate students in state-of-the-art teaching techniques, run by the MIT Teaching and Learning Lab.
\end{itemize}

\section{Service}
{\bf Leadership}
                \begin{itemize}[leftmargin=*] %\itemsep -2pt
                \item {\bf President}, {\it Middle East Education through Technology's student group at MIT} \\ Serving as an ambassador for the MEET program on campus, and recruiting MIT students as summer instructors. \hfill Fall '13 - present
		\item {\bf EdTech Reading Group Co-Organizer}, {\it MIT} \hfill Fall '12 \\ Formed a reading group for MIT students, faculty, and staff to discuss papers relevant to the growing interest in technology in education and education at scale.
                \item {\bf Vice-President}, {\it Eta Kappa Nu, MIT Chapter} \hfill Spring '08 - '09 \\ MIT's EECS honor society 
%\begin{flushright}
%\end{flushright}      
%\item Informally mentor young women wrestlers in Massachusetts
%\item Member of the Board of the Massachusetts Chapter of USA Wrestling
%\item Appointed to be a part-time assistant coach for the new women's wrestling team at Springfield Technical Community College during the Fall '12 season
\end{itemize}

{\bf Program Committees}
 \begin{itemize}[leftmargin=*]
\item ACM Computer-Human Interaction Works-in-Progress ({\bf CHI WiP}) \hfill Jan. '15
\end{itemize}

		 {\bf  Committee Memberships}  
                 \begin{itemize}[leftmargin=*] %\itemsep -2pt
\item {\bf EECS Department Education Committee}, {\it MIT} \hfill Dec. '06 - Fall '08 \\
Served as a student representative during a significant department-wide curriculum redesign.
\item {\bf MIT Council on Educational Technology} \hfill Spring '05 
%\begin{itemize}
%	\item Oversees EECS curriculum
%\end{itemize}
 \end{itemize}
		 


          

		
		 


 
 
% {\bf 2001-2004 Intel International Science and Engineering Fair Participant} \hfill Sept. '00 - May '04 
% \begin{itemize} \itemsep -2pt  % reduce space between items
%\item \textbf{Brain-Computer Interface for the Muscularly Disabled} (2003, 2004): Designed signal processing method for improved performance of existing BCIs.
%	\item \textbf{Speech Imitation through Analysis, Synthesis, and Optimization} (2002): Software intended as a step towards a speech therapy tool for children.
%	\item \textbf{Signal Dissection by Repetitive Smoothing and Extraction} (2001) 
% \end{itemize}
 


 
%		 
%		{\bf Leadership} 
%                \begin{itemize} \itemsep -2pt
%                \item Vice-President, MIT Chapter of Eta Kappa Nu, an EECS honor society 
%\begin{flushright}
%Apr. '08 - Apr. '09
%\end{flushright}      
%		 \end{itemize}
		
 
		

		 

%		 
%		 				 \section{Technological Public Service}
%\begin{itemize}
%\item Software developer for Assured Labor \hfill Spring 2008
%\begin{itemize}
%	\item Helped produce an early prototype of a cell-phone-centered job marketplace as specified by a start-up hoping to improve the lives of workers in developing countries using mobile phones.
%\end{itemize}
%	\item Member of GlobalHealth, a student-led group focused on improving access to health care in Pakistan \hfill Winter, Spring 2007
%\begin{itemize}
%	\item Helped produce a proposal for a computer analysis-based system for tracking and anticipating outbreaks of various diseases in Pakistan. 
%	\item Proposal was granted funding by the Pakistani government.
%\end{itemize}
%\end{itemize}
 
\section{Selected Press}
\begin{itemize}[leftmargin=*] \itemsep -2pt
\item Co-Authored WIRED article: ``MIT Computer Scientists Demonstrate the Hard Way That Gender Still Matters'' \hfill Dec. '14
\item Appeared in \textit{Science}: ``Rising Stars'' (30 May 2003), \textit{Science} 300 (5624), 1368d.
\item Profiled on CNN's \textit{Lou Dobbs Tonight}, in a segment titled ``America's Bright Future''  \hfill Fall '03 
\item Guest on CNN's \textit{American Morning} \hfill May '03
\end{itemize}

\section{Outreach}
\begin{itemize}[leftmargin=*] \itemsep -2pt
\item Reddit AMA with Jean Yang and Neha Nerula, on behalf of MIT CSAIL \hfill Dec. '14
\item Guest speaker for MIT CSAIL's Hour of Code event \hfill Dec. '14
\item Mentor for Harvard Women in CS's ``Women Engineers Code Hackathon'' \hfill Dec. '13
\item New Hampshire TechFest, Agile robotics booth host \hfill Nov. '11
\item Cambridge Science Festival, Agile robotics booth host \hfill May '11
\item MIT Women's Technology Program \hfill July '08, '11 \\
Guest spoke twice to high school girls interested in EECS
\end{itemize}

\section{Other interests and activities}
{\bf Wrestler}
\begin{itemize}[leftmargin=*] \itemsep -2pt
\item Team Member, MIT's NCAA Div. III Varsity Wrestling Team \hfill Winter '08 - '09
\item Competitor, US and Canada in regional \& national women's tournaments \hfill '09 - '12
\item Two-time Training Camp participant, US Olympic Training Center in Colorago Springs, CO \hfill Aug. '10, Sep '12
\item Board member of the Massachusetts Chapter of USA Wrestling \hfill 2012


\end{itemize}

\end{resume} 
 

\end{document}


