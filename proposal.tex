%2. BODY OF THE PROPOSAL
%The body of the proposal must be well organized and phrased in good English, but the specific 
%arrangement and details of coverage can vary depending on the problem, your preferences, and 
%those of your research supervisor. Each proposal, however, should include at least the following 
%topics. 
%i. A brief summary of the background of the problem up to the present must be included 
%giving evidence by reference or otherwise that you have become familiar with this 
%background. In discussing the present status of the problem, you should make evident 
%the extent to which the proposed solution is novel and/or an improvement.
%ii. The probable procedure must be outlined -- from start to finish -- showing which steps 
%are doubtful and therefore subject to change. Include a time schedule that either specifies 
%dates by which various parts of the work should be completed or else allocates a certain 
%number of hours to each major part of the thesis task, such as preparation of samples, 
%experimental work and analysis, correlation and interpretation of results, and preparation 
%of the report. (The total number of hours assigned to a Master's or EE/ECS thesis is 
%nominally 360 hours.)
%iii. A list of the principal equipment and facilities needs must be included, together with 
%the places, which will supply these needs. You should have some assurance that what 
%you need will be available at the time desired. For non-experimental theses, the sources 
%of data (if any) must be stated and the availability of the data assured.
%iv. A bibliography must be included.

